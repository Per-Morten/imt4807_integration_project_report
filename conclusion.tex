\section{Future Work}
While the current demo is functioning, it leaves a lot to be desired. Work with this project will continue over the summer in order to improve several aspects. The plan is to finish the transition over to a pure Javascript client, removing the Unity WebGL solution that is currently being used. Additionally, the product needs to be made more robust, error safe, and user-friendly. The target is to reach a level where it is actually presentable to customers.
Following this is the need to move away from YouTube as a streaming platform, and over to middleware solutions, so as to not continue to break YouTube's terms of service.
It is also desirable to continue the endeavors of optimizing the streaming latency and processing overhead. Additionally, image capture offloading solution mentioned earlier will be tested.

\section{Conclusion}
This paper has presented the process of creating a solution in which a video camera in a Unity application can stream videos to clients, who can interact with the collaborative environment, with a latency of two to three seconds. Within the solution, the client acts as a thin-client and does not require any software installation. Additionally, an overview of alternative solutions that were considered has been provided. In terms of innovativeness, the proposed solution is innovative to a low degree.

Throughout this project, we have gained experience with streaming platforms and related technologies, as well as experience in working with an external partner, which is useful for our future careers. Following, we have also gained experience with technology assessment approaches, to ensure our approach was more systematic, leading to a more satisfactory solution. Through the course, we have also gained experience with the topic of innovation and the process behind it, although due to the nature of the project, the theory was not as applicable.
In conclusion, this has been an interesting and informative project which has furthered our knowledge within the field.
