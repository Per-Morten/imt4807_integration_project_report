\section{Introduction}
Virtual Reality (VR) can be described as \rephrase{computer generated simulations which a person can interact with in a physical way}\cite{vr_definition}. This technology combined with the internet opens up new opportunities for collaboration, allowing users to exist within the same virtual environment, regardless of real life physical distance. \todo{Write quick about collaborative software.} However, VR technology is still quite expensive, and giving each employee within a company their own VR headset to participate in collaborative endeavours is not feasible. As a result of this, alternative solutions allowing users to participate within collaborative software by the use of non-VR technology are provided by several of the software providers\todo{Add cite to competitors?}. However, these solutions often work akin to online multiplayer functionality in games, requiring the users to have hardware capable of running the application, and to install said application on their device. Additionally, it is uncertain how these applications scale with the number of users participating, as more users require more synchronization overhead in the application. This paper presents a prototype solution where users can participate in a limited interaction spectator mode within a VR collaboration application developed by the Norwegian Extended Reality (XR) company Vixel.

This document will first present the case supplied by Vixel, existing similar solutions, and the requirements for the software. \todo{Write rest when finally decided on structure}


\begin{itemize}
    \item Introduction to Vixel
    \item Introduction to VR (Basic)
    \item What is the problem
    \item Related work (sub: Mention that we are unaware of similar WHOLE solutions)
    \item Document Structure
\end{itemize}