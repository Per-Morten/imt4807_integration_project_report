\section{Introduction}
Virtual Reality (VR) can be described as a computer-generated simulation in which a person can interact within a physical manner\cite{vr_definition}. This technology combined with the internet opens up new opportunities for collaboration, allowing users to exist within the same virtual environment, regardless of real life physical distance. Several software providers are currently creating collaborative VR platforms, where users can meet and collaborate on various tasks within a virtual environment. However, VR technology is still quite expensive, and giving each employee within a company their own VR headset to participate in collaborative endeavors is not feasible. As a result of this, alternative solutions allowing users to participate within collaborative software by the use of non-VR technology are provided by several of the existing solutions\cite{rumii_vr, insite_vr}. However, these solutions often work akin to online multiplayer functionality in games, requiring the users to have hardware capable of running the application and to install said application on their device. Additionally, it is uncertain how these applications scale with the number of users participating, as more users require more synchronization overhead in the application. This paper presents a prototype solution where users can participate in a limited interaction spectator mode within a VR collaboration application developed by the Norwegian Extended Reality (XR) company Vixel.

This document will first present the case supplied by Vixel, existing similar solutions, and the requirements for the software. Following, the methodology and high-level architecture will be presented, before per component technology assessment is discussed. Subsequently, the results, innovation, and discussion are presented, before future work and conclusion closes the document. 