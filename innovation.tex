\section{Innovation}
Through this section, we will discuss the level of innovation of both the project as a whole and the more novel component solutions. This is done using the Norwegian Research Council's scale of innovativeness\cite{nrcs}.
The fact that we did not come up with the original idea limits our ability to assess the innovation process behind that solution. Because of this, we primarily focused on some of the more novel design decisions, as well as the project as a whole. We chose not to categorize inventiveness as in our case there was too much overlap between the level of inventiveness and innovation.

\subsection{Overall Solution}
The overall solution could be seen as relatively innovative. On the Norwegian Research Council's\cite{nrcs} scale of innovativeness we argue that it can be categorized as level 3, being innovative for the company, but small compared to the industry.

Normally spectator functionality is integrated into the game or application itself, leading to software requirements; in that you need to install the application, and also the need for hardware capable of running the software. The lightweight project solution has relaxed these normally substantial requirements, as the user does not need to install the client application. Additionally, only a video stream is required, which most normal computing devices can handle without issues. The innovative aspect of this solution is the removal of spectating as a feature tied to the application or game, replaced by an online stream. An additional benefit of this is that less synchronization is required within the game or application, as spectators are only receiving a video feed, at the cost of individual camera control.

The solution allows the clients to interact with the video stream in minor ways. This is similar to functionality found in the "twitch plays series"\cite{twitch_plays}, but the interaction with the stream happens through clicking on the video, rather than through chat. We could also draw parallels to cloud gaming, however, in our case there are multiple users interacting with the same application, rather than just one, and with more restrictions on the interactive options. In these aspects, the client's interface can be seen as innovative.

These aspects are important for the company as it gives them access to more customers, and potentially new customer groups. The solution allows customers without VR hardware to participate in the software, increasing the potential number of users. Additionally, not requiring any software installation makes it much easier to demonstrate the product for potential customers.

We cannot argue that the entire solution is a major improvement for the industry. After an overview scan of competing VR collaboration tools, we identified that several of them had the option of spectating or collaborating through non-VR platforms using an application. This leads us to believe that the video stream approach is fairly innovative, however, we cannot guarantee that others in the industry have not made a similar solution.

\subsection{Per Component}
\subsubsection{Invisible Unity Overlay}
The invisible Unity overlay for the client can be seen as a level 2 innovation, meaning that it is a minor innovation for the company\cite{nrcs}. 

We have seen a similar problem of creating a transparent Unity WebGL application been discussed before\cite{unity_transparent_webgl}, in this case, a partially invisible overlay was desired, in order to create graphical widgets on top of a video stream\cite{unity_forum_partially_invisible}. In our case we only needed input functionality from Unity, meaning that the entire application could be rendered invisible, which is the solution we went with. Using Unity WebGL as an invisible overlay could be seen as innovative, as we assume this is not common practice. However, it is a very heavyweight solution, making the fact that it is not common practice, quite reasonable. As mentioned in Section \ref{par:client_final}, this solution allowed us to prototype and develop faster, making it valuable.

\subsubsection{Using Streaming Platforms as a Distribution Service}
The normal use-case of stream distribution services is to create own solutions or make use of third-party services. In our case, we went for the somewhat different approach of using YouTube and Twitch.tv. We consider this choice as innovation level 2, minor innovation for the company\cite{nrcs}. We are unsure exactly how to categorize the solution based on its prominence in the industry. The reason being that it might go against the terms of service for the different platforms, meaning that due to legal ramifications any similar usage patterns are probably not discussed loudly in the industry. 
